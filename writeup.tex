\documentclass[11pt, oneside]{article}   	% use "amsart" instead of "article" for AMSLaTeX format
\usepackage{geometry}                		% See geometry.pdf to learn the layout options. There are lots.
\geometry{letterpaper}                   		% ... or a4paper or a5paper or ... 
%\geometry{landscape}                		% Activate for rotated page geometry
%\usepackage[parfill]{parskip}    		% Activate to begin paragraphs with an empty line rather than an indent
\usepackage{graphicx}				% Use pdf, png, jpg, or eps§ with pdflatex; use eps in DVI mode
								% TeX will automatically convert eps --> pdf in pdflatex		
\usepackage{amssymb}
\usepackage{natbib}

\newcommand\aj{AJ}                   % Astronomical Journal (the)
%SetFonts

%SetFonts


\title{A Distance Measurement to NGC 3923 Using a Type Ia Supernova}
\author{K. Azalee Bostroem}
%\date{}							% Activate to display a given date or no date

\begin{document}
\maketitle
\section{Introduction}
Measuring distances to astronomical object is challenging. 
The distance to nearby objects can be measured by observing the apparent movement of the object against unmoving, more distant background objects over the course of a year.
This geometric distance is the most reliable, however, it can only be used for objects in our galaxy, out to $\rm{\sim}$1000 parsecs.

To determine the distance to objects outside of our galaxy, we rely on standard candles, objects whose intrinsic brightness can be determined by an observable, such as the peak brightness of a supernova (SN) or the period of variability of a Cepheid star. 
By knowing the intrinsic brightness and the apparent brightness, the distance can be determined.
Type Ia SNe (SNe Ia) are standardizable candles, meaning that they can be calibrated to be standard candles.
SNe Ia are extremely bright explosions of white dwarf stars in binary systems, that have surpassed the Chandrasekhar limit (1.4 $\rm{M_{\odot}}$) through either mass transfer from a red giant companion or through a merger with a white dwarf companion.
Due to the degenerate nature of white dwarfs, when the Chandrasekhar limit is exceeded, overcoming the electron degeneracy pressure, increase in temperature during the collapse is not halted by a corresponding change in pressure.
This produces a runaway nuclear reaction that explodes the star as a SNe Ia, rapidly brightening on the sky and then slowly fading.
The fact that SNe Ia are produced by thermonuclear explosions in white dwarfs with the same mass causes their light curves (brightness over time) to be nearly identical.
For this reason, if we can calibrate the intrinsic brightness of one, we can measure the distance to any SNe Ia. 

This calibration of the intrinsic brightness is performed using the distance ladder. 
A geometric distance is used to calibrate the intrinsic brightness of Cepheid variable stars.
Cepheid variable stars are then used to measure the distance to nearby galaxies that have hosted SNe Ia, allowing for their intrinsic brightness to be calculated.
SNe Ia have an extremely bright ($\rm{M_{r} = -19.03 \pm 0.01}$; \citet{2010folatelli}) peak brightness and can be used to measure the distance to galaxies at $\rm{z>2}$.
These measurements lead to the discovery of dark energy.

In this report, we use SN 2018aoz, a SN Ia, to measure the distance to its host galaxy, NGC 3923.
In Section \ref{LC} we determine the light curve of 2018aoz and in Section \ref{Dist} we use its peak brightness to determine the distance to NGC 3923.
We conclude in Section \ref{conclusion}.

\section{Light Curve Determination}\label{LC}
Aperture photometry was performed on {\it r}-band images provided by Stefano Valenti (private communication) using {\tt photutils} \citep{2018bradley}. 
We fit a 2D gaussian to the SN in each image and determine the median full width half maximum (FWHM) to be 6 pixels. 
We use an aperture of 18 pixels, three times the FWHM.
An annulus with an inner radius of 20 pixels and an outer radius of 24 pixels was used to determine the background.
The background flux was scaled to the science aperture area and subtracted from the SN flux.

Instrumental magnitudes were converted to apparent magnitudes using stars in the image that are also in the APASS catalog \footnote{https://www.aavso.org/apass}.
Aperture photometry was performed on these stars using identical parameters to the SN photometry.
The {\it r}-band zeropoint was then calculated for each observation as the sigma clipped mean and the error as the sigma clipped standard deviation.
Points with errors greater than 0.5 magnitudes were excluded from the dataset.
The light curve is plotted in Figure \ref{fig:LC}.
\begin{figure}
\includegraphics{light_curve}
\caption{The light curve of 2018aoz (red circles). 
The fit to the light curve is also plotted as a black line.\label{fig:LC}}
\end{figure}

\section{Distance to NGC 3923}\label{Dist}
To determine the distance to NGC 3923, we must now find the peak apparent magnitude.
We find a peak magnitude of 12.93 by fitting the light curve with a third order spline.
Using an absolute magnitude of -$\rm{M_{r} = -19.03 \pm 0.01}$ \citep{2010folatelli} and correcting for galactic extinction, $\rm{A_{r}=0.228}$ we find a distance modulus of 
\begin{equation}
\mu = m_{r}-M_{r}-A_{r} = -31.73\pm0.01
\end{equation}
where $\rm{m_{r}}$ is the apparent {\it r}-band magnitude, $\rm{M_{r}}$ is the absolute {\it r}-band magnitude, and $\rm{A_{r}}$ is the {\it r}-band extinction of the Milky Way.
We use the distance modulus equation:
\begin{equation}
d = 10^{\frac{\mu+5}{5}}
\end{equation}
where d is the distance in parsecs and $\rm{\mu}$ is the distance modulus, to determine a distance of $\rm{22.26 \pm 0.10}$ Mpc to NGC 3923.
This value is not corrected for color or decline rate as such analysis is beyond the scope of this report.

\section{Conclusion}\label{conclusion}
In the report we determine the distance to NGC 3923 using SN 2018aoz as a standard candle.
We find the brightness of the SN over time. 
Fitting the light curve with a cubic spline, we find a peak brightness of 12.93, yielding a distance to NGC 3923 of $\rm{22.26 \pm 0.10}$ Mpc.

\bibliography{references}
\bibliographystyle{plain}

\end{document}  